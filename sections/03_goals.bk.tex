\section{Design Tasks of \gaia{}}
\todo{I am thinking about removing this section.}

% Infographics can be represented with vector graphics, like SVG files. 
% Infographics of the same type, such as a bar chart, can exhibit significant variations in their internal SVG structures due to the usage of <g> elements.
% This happens even for SVGs that look very similar.
% However 
% In spite of this diversity, infographics still share similar patterns semantically.

% \begin{enumerate}

% \item \textbf{G1 Expressivity.}
% Users can build complex animation easily via intuitive grammar with high expressive characteristics.
% There is a trade-off between expressiveness and intuitiveness. 
% The former requires language provides low-level operations, which increases the difficulty of writing and reading. 
% This can be alleviated via encapsulation \cite{gganimate}.
% Inspired by this idea, \gaia{} has an intuitive grammar without losing expressivity.

% \item \textbf{G2 Reusability.}
% Users can reuse animations across similar infographics instances or components, even though they might vary in structures or attributes.
% This can avoid duplication of creation and accelerate the creation process.

% \item \textbf{G3 Extensibility.}
% Users can create new templates by composing existing templates without any concrete instance provided.

% \end{enumerate}

\gaia{} is designed to build animated infographics efficiently and accurately.
Based on the goals introduced in \autoref{sec:intro}, we formalize the design task of \gaia{} as:

\begin{enumerate}

\item \textbf{T1 Provide intuitive high-level grammars for animation description (G1).}
Some languages, like transition operators in D3, use low-level expressions to create animations. 
Instead, \gaia{} is close to natural descriptive language and provides a high-level grammar that is independent of the underlying implementation.
It is self-descriptive as well, i.e., “what you see is what you get”.

\item \textbf{T2 Support hierarchical animation/template creation (G1).}
Complex animations can be created by concatenating several sub-animations according to different schedule strategies, such as launching one by one. 
Inspired by the composition design pattern in OOP, \gaia{} allows users to create such animation via composing other animations. 
%This is also straightforward due to the hierarchy of infographics.

\item \textbf{T3 Allow reusing animations across different instances (G1, G2).}
An animation can be encapsulated as a template and be applicable across visualization instances. 
For example, animations designed for bar chart should be applicable to various instances of bar charts. 
In \gaia{}, animations and templates actually share the same grammar since an animation can be regarded as a template for a concrete animated instance.

\item \textbf{T4 Enable parameterization of template usage (G2).}
Templates can be customized so as to apply to similar or repeating components in one visualization instance. 
For example, a common animation template for chart axis can be applied on both x-axis and y-axis in a bar chart instance with adjustment of target or parameters.

\item \textbf{T5 Support template extension (G3).}
Similar to animation creation, \gaia{} enables users to create templates by combining other templates. 
However, it doesn't require any concrete visualization instances. 
Instead, \gaia{} provide a target abstraction for this task. 
The new templates can be used as other existing templates.

\item \textbf{T6 Provide modularity mechanism (G2, G3).}
Due to the diversity of types of infographics and animations, conflicts will become a big issue while creating. 
For instance, animations having the same name but designed for different target types may be used in one spec. 
Hence, module abstraction is introduced to manage spec resources and make sure specs from different users can work well together.

\end{enumerate}
