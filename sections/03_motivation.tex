\section{Motivation and Design Considerations}
\label{sec:motivation}

% [18, 55] katika crowdsourcing community-based approaches

\gaia{} is a high-level declarative language that enables rapid specification of animated infographics.
Taking a static SVG as input, \gaia{} compiler generates a playable animated infographics instance according to the \gaia{} specification.
As mentioned in \autoref{sec:intro}, \gaia{} is designed to be expressive, reusable, and extensible.
Inspired by Katita~\cite{jahanlou2022katika}, animations in \gaia{} are also treated as \textit{independent entity} and can be reused in one instance or across different instances (\ie artworks in Katika).
Following the design of Gemini~\cite{kim2020gemini} and Canis~\cite{ge2020canis}, \gaia{} \textbf{manage animations in a hierarchical structure}, providing constraints that describe the relationship between animations.

Such an animation structure allows some composite animation designs (\eg the animation of ) to be reused within a single instance, but not between different instances.
Because the internal structure and elements of infographics vary from instance to instance.
Unlike some interactive tools (\eg Katika), the input SVG of \gaia{} can be created by any upstream tools (like D3~\cite{bostock2011d3}, Vega-Lite~\cite{satyanarayan2016vega}, Adobe Illustrator ~\cite{AdobeAI}, \etcns).
Canis provides a possible solution to bind roles to elements: set ids, classes, and data to SVG elements and generate dSVG.

This motivates us to design a \textbf{tree-like target abstraction} for \gaia{}.


Then unstructured SVGs need to be mapped to the target abstraction to enable reusability.

To sum up, 
