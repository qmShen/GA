\section{Discussion}
\label{sec:discuss}

\gaia{} is the first declarative grammar for animated infographics, which provides capabilities of flexible expression and reusability.
In this section, we discuss the design considerations, possible usages and limitations of \gaia{}.

\subsection{Cognitive Dimension}
The design of \gaia{} takes into account the Cognitive Dimension~\cite{blackwell2001cognitive}.
First, a user might act in different roles when using \gaia{} (as we mentioned in \autoref{sec:workflow}), so the consistency of the grammar is a concern.
\gaia{} uses the same grammar for both animation and template.
Consistent with the findings in the user study, this significantly reduces the cost of creating and using a template.
Besides, the tree-like spec logic for animations and target types is similar. 
Keeping some attributes consistent, such as ref, also promotes understanding.
Second, the nesting animation of \gaia{} provides a progressive learning style and supports \aniunit{} level evaluation. 
For \aniclass{} authoring, users can design a simple animation first, and then gradually add more details, each time the \aniunit{} can be evaluated immediately.
In user studies, most participants use this approach, reviewing the results as they make changes.
Third, users have the freedom to control their work steps. 
For a complex animation, users can break it down into multiple specs (\aniclass{}). 
Each spec can be evaluated individually and referred by a top-level animation so that the entire project can be completed incrementally, as stated in E1's feedback.

\subsection{Trade-off of Target Abstraction}
\gaia{} decouples the infographics creating and animation designing and employs a virtual layer for reusability.
However, users need to clarify the target type and understand the virtual model before designing animations, which might require more effort in some cases.
We determined that although the virtual layer introduces extra learning costs and usage burdens, these are outweighed by the convenience of reusing templates from professional designers, which is a relatively harder part compared to declaring the types.
In addition, virtual target models bring expert-designed animations from one instance to a class of instances, helping to express, encapsulate, and transfer domain knowledge, creating a more prosperous community environment.
As another insight, the design of the virtual target model (\eg list in \gaia{}) can be used to eliminate the complexity of real structures (as P6 said).
Finally, in the case study, we also show that \gaia{} can fall back to the original SVG structure when the virtual layer is not needed. 

\subsection{Potential Usages}
\gaia{} has the potential to be used in many more scenarios.
\gaia{} is a high-level spec that can be easily read or written directly, but it can also be used as a part of interactive tools for animated infographics authoring (E2, E3). 
Benefiting from the reusability of \gaia{}, beginners will be able to design professional animations faster and their designs can also contribute to the tools' ecosystems.
Even without exposing the reuse concept to the user, tools can take advantage of \gaia{}'s expressiveness and animation library to get the job done.
Furthermore, \gaia{} offers a well-organized format for representing both infographic instances and animations, which can obtain benefits from generative AI models. 
For instance, tools can incorporate AI to aid in animation creation with \gaia{} as an abstraction layer (E2), swiftly beginning with blank projects or carrying out design completion. 
It can even be employed to construct end-to-end generation tools for animated infographics, which is our ultimate objective and further work.
For now, the prototype of \gaia{} has already been used in some research projects for animated infographics authoring and attracted some enterprise engineers for further cooperation.

\subsection{Debugging}
Debugging declarative grammar is challenging~\cite{satyanarayan2015reactive}, and \gaia{} is no exception.
This issue is also mentioned by some participants in the user study.
With our experience working with downstream users, problems (\eg wrong syntax, empty selection and wrong parameters), will consume a lot of energy.
We implemented \gaia{} with these issues in mind, providing a complete logging system and error-handling flow. 
In addition, information related to the animation logic (such as selected elements, final animation structure, computed durations of each animation) can be obtained by the resolved animation object to facilitate debugging. 
It can also be used by developers of upper-level tools for visual animation and debugging.
Even so, we found that new users of \gaia{} still often ran into problems and struggled to get effective tips for changes. 
Smoothing the development pipeline and debugging approaches for \gaia{} is a direction worthy of further study.

\subsection{Limitation and New Opportunities}
\gaia{} introduces the target abstraction, then decouples the work for creators, designers and developers.
However, this workflow has not been fully validated.
The evaluation of expressivity is also demonstrated through the repetition of examples, rather than through the animation result implemented by professional designers.
In addition, we did not go into depth and validate the design of the target type spec and binder spec.
P5 mentioned the generalization of target types and corresponding templates can be further explored.
In expert interviews, E2 shared his experience in creating generic templates, but further research was lacking.
In addition, P5 also asked whether binder specs can be generated automatically or type binding can be done interactively.
These are the directions that can be further studied in the future.