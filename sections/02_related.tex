\section{Related Work}

In this section, we discuss previous work related to \gaia{} from 3 aspects: infographic and animation for infographics, visual animation authoring tools and programming approaches for animation authoring.

\subsection{Infographic and Animation for Infographics}
% infographics
The number of work on infographics published in recent years has increased.
Infographics combine a variety of data-encoded elements and embellishments, which can attract readers’ interests easily~\cite{bateman2010useful, haroz2015isotype, borkin2013makes, wang2019emotional}. % from InfoMotion [BMG∗10, HKF15, BVB∗13, WSK∗19]
% hierarchy; Visual Information Organization
Some existing researches attempt to understand infographics from semantic meanings of visual elements~\cite{lu2020exploring, wang2021animated}. 
Lu \etal~\cite{lu2020exploring} gave a concept of visual information flow. They pointed out one infographic can contain multiple visual groups that share similarity. 
These groups are laid out to follow a backbone and form a hierarchy of infographics naturally.
Wang's work~\cite{wang2021animated} also illustrated this insight existed in a large range of real infographics.
They named such groups as repeating units and pointed out that different visual elements in a group can have different roles semantically.
% nested hierarchy
Nested structure in the element hierarchy is common and can be easily created via some DSLs like D3~\cite{bostock2011d3} and Protovis~\cite{heer2010declarative}.
In the work of Liu \etal~\cite{liu2018data}, visual elements are arranged by collections, which can be nested for organization.
For example, some infographics contain several stacked bars, which form a multi-layered hierarchy: stacked bar groups and rectangles in each group.
% relation to Gaia
In \gaia{}, we follow the intuition of nested hierarchy in infographics to build an abstraction of animated targets. 

% animated infographics
Animated infographics are regarded as an efficient tool that can express an informative story using intuitive and attractive effects applied on visual elements~\cite{blazer2019animated, brehmer2016timelines}.
Researchers have investigated a wide range of cases from the real world and conducted surveys for understanding and creation~\cite{wang2021animated, shi2021communicating}. 
% InfoMotion
By collecting animations of infographics from multiple sources, like PowerPoint templates and video sites, Wang \etal~\cite{wang2021animated} summarized the design space of animated infographics. 
They claimed that effects can be applied to a group of elements, then be organized via three arrangement styles.
\gaia{} borrows the primitives of timing arrangement and adapts them to animations with a more complex timeline.
% Nan's survey
Some other work focus on effects used in animations~\cite{shi2021communicating}.
They derived animated visual narratives that fall into 8 categories of narrative strategies. 
These work inspire the design of animations in \gaia{}, but with the type system, \gaia{} can design animations for a semantic group of elements by combining these primitive effects.

\subsection{Visual Animation Authoring Tools}

Creation of animated infographics is a challenging task~\cite{jahanlou2022katika, howlong}, so a few visual animation authoring tools are proposed.
% Keyframe-based tools
Keyframe-based tools, like Adobe After Effects~\cite{AdobeAE} and Data Animator~\cite{thompson2021data}, provided low-level operations for motion creation.
They allow users to specify the properties of visual elements on each keyframe and create intermediate frames according to the easing function.
Besides, it is time-consuming when creating or modifying visual properties and time settings on a set of marks.
\gaia{} doesn't need to define keyframes. Instead, high-level grammar is employed for easy reading and writing.

% Template-based
Template-based tools, as another category, provide a set of preset templates to facilitate animation creation.
PPT~\cite{msppt}, DataClips~\cite{amini2016authoring} and Flourish~\cite{flourish} fall into this category.
In these tools, animations can be created intuitively and easily by applying templates on targeted elements instead of adjusting low-level properties. 
However, most of them only provide a small number of effects and can't be extended, which limits their applicability. 
Besides, their flexibility relies on pre-defined templates, thus are less expressive.
As such, creating complex animations with them is difficult, even impossible.
Data Player~\cite{shen2023dataplayer} interweaves data video authoring with narrations, providing a set of pre-designed animation presets.
Katika~\cite{jahanlou2022katika} provides an end-to-end solution for creating animated graphics visually and also provides a motion library that can be extended by the community. 
Although it simplifies the creation pipeline, Katika can only employ statics graphics (called artworks in Katika) from its own embedded library.
% Gaia
\gaia{} also uses a concept of templates and allows users to extend them smoothly.
As a high-level grammar, \gaia{} and its compiler can be integrated as a part of interactive tools for animated infographics authoring.
We refer interested readers to the survey on visualization authoring methods conducted by Grammel \etal~\cite{grammel2013survey}.

\subsection{Programming Approaches for Animation Authoring}
To create animation more accurately and easily, some designers prefer programming approaches, which can be categorized into low- and high-level languages.
% low-level
Low-level languages, such as processing~\cite{processing} and D3~\cite{bostock2011d3}, provide APIs for low-level operations that enable users to control the properties of animated targets directly.
For example, D3 uses a transition operation that enables users to indicate what and how visual properties (\eg width and color) change on selected elements.
Due to their high expressiveness, these languages have gained widespread use for creating custom and informative animations.
However, they require users to have experience in graphic editions and animation creation, resulting in a steep learning curve. 
Users need to specify when and how these properties change for every element in detail, which is non-intuitive and time-consuming for complex animations with many animated elements.

% high-level
High-level languages are typically preferred as they favor conciseness over expressiveness~\cite{mcnutt2023nogrammar}. 
% json-based
Some languages adopt JSON to express animations at a high level~\cite{zong2022animated, kim2020gemini, kim2021gemini, ge2020canis, ge2020canis, ge2021cast}. %[Animated Vega-Lite, Gemini, Gemini2, Canis, CAST, Lottie]
Gemini and Gemini2~\cite{kim2020gemini, kim2021gemini} focus on transitions between two static visualizations. 
Animated Vega-Lite~\cite{zong2022animated} is proposed for generating animations and binding logic for interactive graphics. 
% hiding complexity
With hiding under-layered mechanisms, some grammars describe animations as effects applied on selected elements, thus are more intuitive and concise compared to the low-level languages.
For example, Canis and CAST~\cite{ge2020canis, ge2021cast} provide several basic pre-defined effects to create animation without setting the initial and end values of properties.
Via grouping operation, they can bind effects to a set of marks with a given align strategy thus the number of animations defined can be reduced.
While these languages create satisfied animations quickly with concise specifications, they focus on data-driven charts instead of infographics.
Another language, gganimate~\cite{gganimate}, uses concise code to express animations.
Similar to Canis, it also employs encapsulation of animations to hide the details.
However, it is mainly designed to create transitions for a specific chart type and can only be used in the programming language R.

% Gaia
Inspired by these work, \gaia{} is designed with a JSON-style grammar that strikes a good balance between expressiveness and conciseness.
% first difference
\gaia{} is the first one who considers types in animation authoring.
According to our insights, elements in an infographic have different types and form a hierarchy. 
Exiting methods, like Canis, can only process groups of the same marks.
Without awareness of types of animated targets, high-level languages can only provide generic and simple effects (such as fade and wipe in Canis, enter\_fade in gganimate), which makes spec verbose when creating complex animations.
% second difference
% Besides, pre-defined effects set in both Canis and gganimate are hard to extend.
Besides, pre-defined effect sets of existing tools are hard to extend.
McNutt~\cite{mcnutt2023nogrammar} pointed out the value of extensibility in DSLs, but none have provided a satisfactory solution for it in the field of animation creation.
\gaia{} brings extensibility to this kind of grammar for the first time to create new animations or target types in a declarative way and enable reusing, which facilitates creation greatly.